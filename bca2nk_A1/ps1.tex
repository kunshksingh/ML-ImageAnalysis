\documentclass[11pt]{article}
\usepackage{fullpage,url}
\usepackage{amsmath}
\usepackage{graphicx}  

\usepackage[letterpaper,top=1in,bottom=1in,left=1in,right=1in,nohead]{geometry}

\setlength{\parindent}{0in}
\setlength{\parskip}{6pt}

\begin{document}
\thispagestyle{empty}
{\large{\bf CS/ECE-4501/6501  \hfill Due by Sep 14 $11:59:59$PM}}\\

{\LARGE{\bf Problem Set I}}
\vspace{0.2\baselineskip}
\hrule

\begin{enumerate}

\item ($20\%$) Prove the properties of convolution. For all continuous function $f$, $g$, and $h$, the following axioms hold. {\bf Please make sure to lay out all key steps of your proof for full credits.} 
\begin{itemize}
 \item Associativity: $(f*g)*h = f*(g*h)$
 \item Distributivity: $f*(g+h) = f*g + f*h$
 \item Differentiation rule: $(f*g)' = f'*g = f * g'$
 \item Convolution theorem: $\mathcal{F}(g*h) = \mathcal{F}(g) \mathcal{F}(h)$, where $\mathcal{F}$
 denotes Fourier transform
\end{itemize}

%% Problem 2
\item ($25\%$) Frequency smoothing: 
\begin{enumerate}
\item Compute Fourier transform of the given image
{\tt lenaNoise.PNG} by using {\tt fft2} function in Matlab (or {\tt numpy.fft.fft2} in Python), and then center the low frequencies (e.g., by using {\tt fftshift}).
\item Keep different number of low frequencies (e.g., $10^2, 20^2, 40^2$ and the full dimension), 
but set all other high frequencies to $0$.
\item Reconstruct the original image ({\tt ifft2}) by using the new generated frequencies in step (b).
\end{enumerate}
{\bf Submit the code and include the restored images with different number of low frequencies in your report.}

%% Problem 3
\item ($55\%$) Implement gradient descent algorithm for ROF model with total variation minimization. {\bf All codes and a two-page report including problem description, a concrete optimization algorithm, and experimental results (a denoised image and a convergence graph that generated by your best-tuned parameters) with discussions should be submitted.}  

NOTE that
\begin{itemize}
\item Test your program with different Gaussian noises ($\sigma = 0.01, 0.05, 0.1$) and include all results in your report. A matlab/python function of 'GenerateNoiseImage' is given for your reference. 
\item The forward / backward difference for computing image gradient is given in {\tt Dx / Dxt}. Feel free to use it or write your own. 
\item A detailed class note of deriving total variation, computing gradient term, and gradient descent algorithm can be downloaded from Collab. 
\end{itemize}
\end{enumerate}

\end{document}
